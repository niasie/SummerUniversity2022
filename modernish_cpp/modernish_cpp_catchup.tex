\documentclass[aspectratio=43, t]{beamer}
\usetheme{CSCS}

\usepackage{fontspec}
\usepackage{unicode-math}
\usepackage{minted}
\usepackage{xurl}
\usepackage{hyperref}

\setminted{autogobble, obeytabs, tabsize=2, breaklines = true}
\setsansfont[Scale = 0.85]{TeX Gyre Heros}

% define footer text
\newcommand{\footlinetext}{Modern(ish) C++ Catch-up}

% Select the image for the title page
%\newcommand{\picturetitle}{cscs_images/image3.pdf}
\newcommand{\picturetitle}{cscs_images/image5.pdf}
%\newcommand{\picturetitle}{cscs_images/image6.pdf}

\author{Michal Sudwoj}
\title{Modern(ish) C++}
\subtitle{Catch-up Session}
\date{13.07.2022}

% \setbeameroption{show notes on second screen = right}
\begin{document}
% disable Pygment error highlighting
\renewcommand{\fcolorbox}[4][]{#4}

\makeatletter
\patchcmd{\beamer@calculateheadfoot}{\advance\footheight by 4pt}{\advance\footheight by 24pt}{}{}
\beamer@calculateheadfoot
\makeatother

\begin{frame}[plain, c]
	\titlepage
\end{frame}

\section*{Contents}
\begin{frame}
	\frametitle{\secname}

	\begin{itemize}
		\item \mintinline{c++}{auto}
		\item Type Constraints (Concepts)
		\item Lambda Expressions
		\item Structured Bindings
		\item Range-Based \mintinline{c++}{for}
		\item Demo / Hands-on: \mintinline{c++}{std::vector}
	\end{itemize}
\end{frame}

\section*{\texorpdfstring{\mintinline{c++}{auto}}{auto}}
\subsection*{Type Deduction (C++11)}
\begin{frame}[fragile]
	\frametitle{\secname: \subsecname\footnotemark[1]{}}
	\footnotetext[1]{\url{https://en.cppreference.com/w/cpp/language/auto}}

	Variable definition: deduce type from definition
	\begin{minted}{c++}
		auto i =  1 ; // `i` is `int`
		auto d =  1.; // `d` is `double`
		auto c = '1'; // `c` is `char`

		auto & j = i; // `j` is `int &`
		auto & k = j; // `k` is `int &`
	\end{minted}
	\footnotetext[2]{\url{https://cppinsights.io/s/6a805cbf}}
\end{frame}

\begin{frame}[fragile]
	\frametitle{\secname: \subsecname}

	Cave: Expression Templates (eg.\ \texttt{Eigen}\footnote[1]{\url{https://eigen.tuxfamily.org/dox/TopicPitfalls.html\#title3}}, \textellipsis)
	\begin{minted}{c++}
		double a;
		Eigen::Vector x, y;

		auto z = a * x + y; // `z` is not `Eigen::Vector`, but something else ...
	\end{minted}
	\footnotetext[2]{\url{https://isocpp.github.io/CppCoreGuidelines/CppCoreGuidelines\#Res-auto}}
	\footnotetext[3]{\url{https://herbsutter.com/2013/08/12/gotw-94-solution-aaa-style-almost-always-auto/}}
\end{frame}

\subsection*{Abbreviated Function Template (C++20)}
\begin{frame}[fragile]
	\frametitle{\secname: \subsecname\footnotemark[1]{}}
	\footnotetext[1]{\url{https://en.cppreference.com/w/cpp/language/function_template\#Abbreviated_function_template}}

	\begin{minted}{c++}
		bool is_positive(auto x) { return x > 0; }
		// is equivalent to
		template<class T>
		bool is_positive(T x) { return x > 0; }

		auto add(auto x, auto y) { return x + y; }
		// is equivalent to
		template<class T, class U>
		auto add(T x, U y) { return x + y; }
	\end{minted}
\end{frame}

\subsection*{Type Constraints (C++20)}
\begin{frame}[fragile]
	\frametitle{\secname: \subsecname}

	\begin{minted}{c++}
		#include <concepts>

		// OK
		std::integral auto i = 1;

		// error: deduced type 'double' does not satisfy 'integral'
		std::integral auto d = 1.;

		// OK
		std::integral auto c = '1';
	\end{minted}
\end{frame}

\section*{Lambda Expressions (C++11)}
\begin{frame}[fragile]
	\frametitle{\secname\footnotemark[1]{}}
	\footnotetext[1]{\url{https://en.cppreference.com/w/cpp/language/lambda}}

	Create an anonymous function, which can capture its environment (\textbf{a closure}\footnote[2]{\url{https://en.wikipedia.org/wiki/Closure_(computer_programming)}})
	\begin{minted}{c++}
		std::vector<int> xs = {1, /* ..., */ 100};
		std::for_each(xs.cbegin(), xs.cend(),
			[](int x) { // this is a lambda
				bool div3 = (x % 3 == 0);
				bool div5 = (x % 5 == 0);
				if (div3 && div5) { std::cout << " fizz buzz,"; }
				else if (div3) { std::cout << " fizz,"; }
				else if (div5) { std::cout << " buzz,"; }
				else { std::cout << ' ' << x << ','; }
			}
		);
	\end{minted}
	\footnotetext[3]{\url{https://cppinsights.io/s/49e154eb}}
\end{frame}

\begin{frame}[fragile]
	\frametitle{\secname}

	\begin{minted}{c++}
		auto add = [](int x, int y) { return x + y; };

		// is __similar__ to

		auto add(int x, int y) { return x + y; };

		// but is **actually equivalent** to

		class Adder { // some unique compiler-generated name
			public:
				auto operator()(int x, int y) { return x + y; }
		};
		auto add = Adder{};
	\end{minted}
	\footnotetext[1]{\url{https://cppinsights.io/s/b8a37260}}
\end{frame}

\subsection*{Captures}
\begin{frame}[fragile]
	\frametitle{\secname: \subsecname\footnotemark[1]{}}
	\footnotetext[1]{\url{https://en.cppreference.com/w/cpp/language/lambda\#Lambda_capture}}

	\begin{minted}{c++}
		int a = 1;
		int b = 2;
		int c = 3;
		// don't capture anything
		auto add = [](int x, int y) { return x + y; };
		// capture `a` **by value**
		auto add_a = [a](int x) { return x + a; };
		// capture `a` **by reference**
		auto inc_a = [&a]() { ++a; };
		// capture what is needed **by value**
		auto add_ab = [=](int x) { return x + a * b; };
		// capture what is needed **by reference**
		auto inc_ab = [&]() { ++a; ++b; };
	\end{minted}
	\footnotetext[2]{\url{https://cppinsights.io/s/f493611f}}
\end{frame}

\begin{frame}[fragile]
	\frametitle{\secname: \subsecname\footnotemark[1]{}}
	\footnotetext[1]{\url{https://en.cppreference.com/w/cpp/language/lambda\#Lambda_capture}}

	We can also create new bindings:
	\begin{minted}{c++}
		// equivalent to `add_a`
		auto add_c = [c = a](int x) { return x + c; };
		// equivalent to `inc_a`
		auto inc_c = [&c = a]() { ++c; };
	\end{minted}
	\footnotetext[2]{\url{https://cppinsights.io/s/f493611f}}
\end{frame}

\begin{frame}[fragile]
	\frametitle{\secname: \subsecname}

	\begin{minted}{c++}
		// given
		int i = 1; char c = '1'; double d = 1.0;
		// then
		auto f = [&, d = d + 1](auto x) { ++i; return x + d; };
		// is equivalent to
		class SomeUniqueCompilerGeneratedName {
			private:
				int & i; double d;
			public:
				SomeUniqueCompilerGeneratedName(int & i, double d) : i{i}, d{d + 1} {}
				auto operator()(double x) { ++i; return x + d; }
		};
		SomeUniqueCompilerGeneratedName f(i, d);
	\end{minted}
	\footnotetext[1]{\url{https://cppinsights.io/s/e2f81bf1}}
\end{frame}

\subsection*{Cave}
\begin{frame}[fragile]
	\frametitle{\secname}

	\begin{enumerate}
		\item since the compiler generates a unique name for each lambda's class, the use of \mintinline{c++}{auto} is required
			\begin{itemize}
				\item the class name is generated by the compiler → not known to the programmer
					\begin{minted}{c++}
						SomeUniqueCompilerGeneratedName f = [](){};
					\end{minted}
			\end{itemize}
		\item each lambda is a unique object
			\begin{itemize}
				\item different class
				\item different instance
			\end{itemize}
	\end{enumerate}
	\footnotetext[1]{\url{https://isocpp.github.io/CppCoreGuidelines/CppCoreGuidelines\#SS-lambdas}}
\end{frame}

\section*{Structured Bindings (C++17)}
\subsection*{Before C++11}
\begin{frame}[fragile]
	\frametitle{\secname\footnotemark[1]{}: \subsecname}
	\footnotetext[1]{\url{https://en.cppreference.com/w/cpp/language/structured_binding}}

	\begin{minted}{c++}
		std::vector<std::tuple<double, double>> points = { /* ... */ };


		for (int i = 0; i < points.size(); ++i) {
			// ugh ...
			double x = std::get<0>(points[i]);
			double y = std::get<1>(points[i]);
			// do something ...
		}
	\end{minted}
	\footnotetext[2]{\url{https://cppinsights.io/s/fc6308a5}}
\end{frame}

\subsection*{Since C++11}
\begin{frame}[fragile]
	\frametitle{\secname: \subsecname}

	\begin{minted}{c++}
		std::vector<std::tuple<double, double>> points = { /* ... */ };


		for (int i = 0; i < points.size(); ++i) {
			// better, but still ugh ...
			double x, y;
			std::tie(x, y) = points[i];
			// do something ...
		}
	\end{minted}
	\footnotetext[1]{\url{https://cppinsights.io/s/9adaa2e8}}
\end{frame}

\subsection*{Since C++17}
\begin{frame}[fragile]
	\frametitle{\secname: \subsecname}

	\begin{minted}{c++}
		std::vector<std::tuple<double, double>> points = { /* ... */ };


		for (int i = 0; i < points.size(); ++i) {
			// nice!
			auto [x, y] = points[i];
			// do something ...
		}
	\end{minted}
	\footnotetext[1]{\url{https://cppinsights.io/s/229ac736}}
\end{frame}

\begin{frame}[fragile]
	\frametitle{\secname}

	Also works for arrays, structs, \textellipsis
	\begin{minted}{c++}
		double p[2] = {1.0, 2.0};
		auto [px, py] = p;

		struct Vector3 {
			double x;
			double y;
			double z;
		};

		Vector3 v{3.0, 4.0, 5.0};
		auto [vx, vy, vz] = v;
	\end{minted}
	\footnotetext[1]{\url{https://cppinsights.io/s/9ffa5aa0}}
\end{frame}

\section*{Range-based \texorpdfstring{\mintinline{c++}{for}}{for} (C++11)}
\begin{frame}[fragile]
	\frametitle{\secname\footnotemark[1]{}}
	\footnotetext[1]{\url{https://en.cppreference.com/w/cpp/language/range-for}}

	\begin{minted}{c++}
		for (auto & point : points) {
			std::cout << point << '\n';
		}
		// is desugared (approximately) to
		{ // new scope to not leak variables
			auto range = points; // handle side-effects!
			auto begin = range.begin();
			auto end   = range.end();
			for (; begin != end; ++begin) {
				auto & point = *begin;
				std::cout << point << '\n';
			}
		}
	\end{minted}
	\footnotetext[2]{\url{https://cppinsights.io/s/a4a0464e}}
\end{frame}

\section*{Conclusion}
\begin{frame}[fragile]
	\frametitle{\secname}

	{\small
		\begin{minted}{c++}
			struct Point { double x, y; };
			std::vector<Point> points = { /* ... */ };
			Point origin = { 0.0, 0.0 };
			auto dist = [origin](double x, double y) {
				auto dx = origin.x - x;
				auto dy = origin.y - y;
				return std::sqrt(dx * dx + dy * dy);
			};
			std::sort(points.begin(), points.end(),
				[&dist](auto a, auto b) {
					return dist(a.x, a.y) < dist(b.x, b.z);
				}
			);
			for (auto & [x, y] : points) {
				std::cout << dist(x, y) << ' ' << x << ' ' << y << '\n';
			}
		\end{minted}
	}
	\footnotetext[1]{\url{https://cppinsights.io/s/f5e9202b}}
\end{frame}

\cscsthankyou{Questions?}

\end{document}
